\chapter{Bevezetés}
\label{chap:intro}

\noindent 
A szakdolgozatom célja egy forráskód elemző eszközkészlet, névlegesen a SourceMeter  bővítése volt, melynek az Analyzer-Javascript alrészébe TypeScript nyelv elemzésének implementálása volt a fő célja.
Mivel ez a szoftvercsomag sok, kisebb programból áll össze, melyek egymásra épülnek, az új nyelv teljeskörű támagotásához több alprogramban kellett változtatásokat hozni. Emellett szükséges volt arra is figyelni hogy a meglévő JavaScript támogatás is megmaradjon.

A SourceMeter eszközkészleten belül több szoftvercsomag található, melyek segítségével különböző nyelvek statikus elemzését lehet végezni. Ezek közül a csomagok közül a SourceMeter-JavaScript csomagnak adott programjain végeztem módosításokat, többek között a JavaScript elemzőn (JavaScript Analyzer, továbbiakban JSAN) és az ESLintRunner nevű programon. 

Ezen bővítések mellett egyéb, működést, karbantarthatóságot és használhatóságot befolyásoló változtatást valósítottam meg, melyeknek köszönhetően az említett programok karbantarthatósága javult, csökkent az eszközkészlet teszteléséhez szükséges idő.

Az eszközkészlet bővítése egyszerre több ember által volt végezve. Mivel a módosított szoftvercsomag alprogramjai egymásra épülnek, fontos volt a csapatmunka és folyamatos, effektív kommunikálás hogy a fejlesztés sikeresen és hatékonyan haladhasson. Néhány helyen a munkafolyamatok során átfedések, és feloszthatatlan feladatokat volt szükséges elvégezni. 