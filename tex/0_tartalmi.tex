\chapter*{Tartalmi összefoglaló}
\addcontentsline{toc}{section}{Tartalmi összefoglaló}

\noindent\textbf{A téma megnevezése:}

\noindent SourceMeter Javascript forráskód elemző szoftvercsomag kiegészítése TypeScript nyelv támogatásával, egyéb tesztelési bővítésekkel.

\noindent\textbf{A megadott feladat megfogalmazása:}

\noindent A feladat folyamán implementálni kell a SourceMeter adott eszközeibe TypeScript elemzésre vonatkozó különböző bővítéseket létrehozni, emellett a tesztelési folyamat kiegészítése új tesztesetekkel, célpontokkal (targetekkel).

\noindent\textbf{A megoldási mód:}

\noindent Az eszközök által használt nyelvi séma bővítése és az érintett szoftverekben új funkciók létrehozása, módosítások implementálása szükséges a feladat végrehajtására.

\noindent\textbf{Alkalmazott eszközök, módszerek:}

\noindent Fejlesztés során forráskód módosításra Visual Studio Code, forráskód fordításhoz Windows rendszereken Visual Studio 2017, Linux rendszereken Unix Makefiles volt használva. Séma módosítására Visual Paradigm volt szükséges. Tesztelések végbevitele lokális rendszeren a fordításhoz használt eszközök, központi szervergépen pedig Jenkins voltak használva.

\noindent\textbf{Elért eredmények:}

\noindent A szoftvercsomag sikeresen tud elemezni TypeScript nyelven írt kódbázisokat. Tesztelés folyamata a SourceMeter különböző szoftvercsaládjainak egyes eszközeire meg lettek szabva, tesztelési folyamat időigénye csökkent.

\noindent\textbf{Kulcsszavak:}

\noindent JavaScript, TypeScript, Statikus kódelemzés, AST, ASG, C, CMake, Python
